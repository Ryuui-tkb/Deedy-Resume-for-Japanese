%%%%%%%%%%%%%%%%%%%%%%%%%%%%%%%%%%%%%%%
% Deedy - One Page Two Column Resume
% LaTeX Template
% Version 1.2 (16/9/2014)
%
% Original author:
% Debarghya Das (http://debarghyadas.com)
%
% Original repository:
% https://github.com/deedydas/Deedy-Resume
%
% IMPORTANT: THIS TEMPLATE NEEDS TO BE COMPILED WITH XeLaTeX
%
% This template uses several fonts not included with Windows/Linux by
% default. If you get compilation errors saying a font is missing, find the line
% on which the font is used and either change it to a font included with your
% operating system or comment the line out to use the default font.
% 
%%%%%%%%%%%%%%%%%%%%%%%%%%%%%%%%%%%%%%
% 
% TODO:
% 1. Integrate biber/bibtex for article citation under publications.
% 2. Figure out a smoother way for the document to flow onto the next page.
% 3. Add styling information for a "Projects/Hacks" section.
% 4. Add location/address information
% 5. Merge OpenFont and MacFonts as a single sty with options.
% 
%%%%%%%%%%%%%%%%%%%%%%%%%%%%%%%%%%%%%%
%
% CHANGELOG:
% v1.1:
% 1. Fixed several compilation bugs with \renewcommand
% 2. Got Open-source fonts (Windows/Linux support)
% 3. Added Last Updated
% 4. Move Title styling into .sty
% 5. Commented .sty file.
%
%%%%%%%%%%%%%%%%%%%%%%%%%%%%%%%%%%%%%%%
%
% Known Issues:
% 1. Overflows onto second page if any column's contents are more than the
% vertical limit
% 2. Hacky space on the first bullet point on the second column.
%
%%%%%%%%%%%%%%%%%%%%%%%%%%%%%%%%%%%%%%


\documentclass[]{deedy-resume-openfont}
\usepackage{fancyhdr}

    
\pagestyle{fancy}
\fancyhf{}
    
\begin{document}

%%%%%%%%%%%%%%%%%%%%%%%%%%%%%%%%%%%%%%
%
%     LAST UPDATED DATE
%
%%%%%%%%%%%%%%%%%%%%%%%%%%%%%%%%%%%%%%
\lastupdated

%%%%%%%%%%%%%%%%%%%%%%%%%%%%%%%%%%%%%%
%
%     TITLE NAME
%
%%%%%%%%%%%%%%%%%%%%%%%%%%%%%%%%%%%%%%
\namesection{戚}{チー}{玉奇}{ヨイチー}{ \urlstyle{same}\href{mailto:whoarei@whoareu.com}{whoarei@whoareu.com} | 080ー8888ー8888
}

%%%%%%%%%%%%%%%%%%%%%%%%%%%%%%%%%%%%%%
%
%     COLUMN ONE
%
%%%%%%%%%%%%%%%%%%%%%%%%%%%%%%%%%%%%%%

\begin{minipage}[t]{0.25\textwidth} 

%%%%%%%%%%%%%%%%%%%%%%%%%%%%%%%%%%%%%%
%     EDUCATION
%%%%%%%%%%%%%%%%%%%%%%%%%%%%%%%%%%%%%%

\section{学歴} 

\subsection{筑波大学}
\descript{修士,コンピュータサイエンス}
\location{GPA: 3.7 / 4.0}
\location{筑波, 日本}
\location{2018.04-2020.03 (見込み)}
\sectionsep

\subsection{浙江大学}
\descript{学士,ソフトウェア工学}
\location{GPA: 3.4 / 4.0}
\location{杭州, 中国}
\location{2013.09-2017.07}
\sectionsep

%%%%%%%%%%%%%%%%%%%%%%%%%%%%%%%%%%%%%%
%     LINKS
%%%%%%%%%%%%%%%%%%%%%%%%%%%%%%%%%%%%%%

\section{リンク}
Github:// \href{https://github.com/vinci7}{\bf vinci7} \\
LinkedIn://  \href{https://www.linkedin.com/in/qiyuqi}{\bf qiyuqi} \\
\sectionsep

%%%%%%%%%%%%%%%%%%%%%%%%%%%%%%%%%%%%%%
%     COURSEWORK
%%%%%%%%%%%%%%%%%%%%%%%%%%%%%%%%%%%%%%

\section{履修科目}
\subsection{修士}
コンピュータネートワーク \\
電子商取引 \\
暗号理論 \\
ソフトウェア工学 \\
英語会話 \\ 
\sectionsep

\subsection{学士}
データ構造とアルゴリズム \\
オペレーティングシステム \\ 
データベース \\
Javaアプリケーション技術 \\
Python開発技術 \\
情報セキュリティ \\
\sectionsep
%%%%%%%%%%%%%%%%%%%%%%%%%%%%%%%%%%%%%%
%     SKILLS
%%%%%%%%%%%%%%%%%%%%%%%%%%%%%%%%%%%%%%

\section{スキル}
\subsection{プログラミング}
\location{5000 行以上}
Java \textbullet{} PHP \textbullet{} Python \textbullet{} Go \textbullet{} Pascal\\
\location{1000 - 5000 行}
C \textbullet{} Javascript \textbullet{} HTML \textbullet{} \LaTeX\ \\
\location{1000 行未満}
C++ \textbullet{} SQL \textbullet{} Shell \\ 
\sectionsep

\subsection{ツール | フレーム}
\location{二年以上}
Git \textbullet{} Django \textbullet{} Vue.js \textbullet{} Codeigniter \textbullet{} Markdown \\
\location{半年以上}
Docker  \textbullet{} SVN \textbullet{} Jenkins \textbullet{} Regex\\
\sectionsep

\subsection{ソフトウエア}
\location{半年以上}
Github \textbullet{} JIRA \textbullet{} WebStorm \textbullet{} GoLand \textbullet{} IntelliJ IDEA \textbullet{} Microsoft Office

%%%%%%%%%%%%%%%%%%%%%%%%%%%%%%%%%%%%%%
%
%     COLUMN TWO
%
%%%%%%%%%%%%%%%%%%%%%%%%%%%%%%%%%%%%%%

\end{minipage} 
\hfill
\begin{minipage}[t]{0.73\textwidth} 

%%%%%%%%%%%%%%%%%%%%%%%%%%%%%%%%%%%%%%
%     EXPERIENCE
%%%%%%%%%%%%%%%%%%%%%%%%%%%%%%%%%%%%%%

\section{職歴}
\runsubsection{東京掲示板グロープ}
\descript{フルスタックエンジニア(アルバイト) }
\location{2019.02 - 現在 | 東京, 日本}
\vspace{\topsep}
\begin{tightemize}
    \item 在日中国人のための情報サービスWebサイトの実装する。
    \item Vue.js、Element.ui、JavaScriptおよびHTMLに基づく。
\end{tightemize}
\sectionsep

\runsubsection{QuLian テクノロジー}
\descript{Golangソフトウェアエンジニア (インターンシップ) }
\location{2018.07 - 2018.09 | 杭州}
\begin{tightemize}
    \item IPCDNというのはIPFS(Distributed Web System)に基づくCDN(Content Distribution Network)を実現するプロジェクトです。
    \item IPCDNのビデオ再生を高速化するためのビデオオンデマンドモジュールを実装する。
    \item 元のアップロードされたビデオをトランスコードするトランスコーダサーバを実装する。
\end{tightemize}
\sectionsep

\runsubsection{Youzan テクノロジー}
\descript{テスト開発エンジニア}
\location{2017.02-2017.08 | 杭州}
\begin{tightemize}
\item Java、Spring、およびJenkinsによるオンラインユーザーインターフェース \\ の自動テストケースの設計および実装。
\item テストカバレッジを改善するためにテストケースを最適化する。
\item テストケースを設計するための思考の枠組みをまとめる。
\end{tightemize}
\sectionsep

\runsubsection{北京小米科技有限責任会社}
\descript{DevOps エンジニア (インターンシップ)}
\location{2015.07-2015.08 | 北京}
\begin{tightemize}
\item クラウドサーバーSDKのGoバージョンを実装する。
\end{tightemize}

%%%%%%%%%%%%%%%%%%%%%%%%%%%%%%%%%%%%%%
%     RESEARCH
%%%%%%%%%%%%%%%%%%%%%%%%%%%%%%%%%%%%%%

\section{プロジェクト}

\runsubsection{{\bf Unitally}}
\descript{オーナー}
\location{2018.2}
\begin{tightemize}
    \item 個人記帳のためのWeChatアプレット、1100人以上のユーザーを獲得する。
    \end{tightemize}
\sectionsep

\runsubsection{{\bf 永楽健康アプリ}}
\descript{学生参加者}
\location{2015.07}
\begin{tightemize}
    \item PHP、CodeIgniter、およびMysqlに基づいて、\\ 研究室の健康管理プロジェクト用のバックエンドAPI開発を実装する。
    \item PythonとBeautifulSoupライブラリに基づいてWebから医療データをクロールする。
    \end{tightemize}
\sectionsep

\runsubsection{{\bf ZJUスポーツアシスタント}}
\descript{オーナー}
\location{2014.10}
\begin{tightemize}
    \item スポーツパンチデータに効率的にアクセスするためのWeChatアプレット。
    \item 3000以上のユーザーを獲得し、ターゲットユーザーグループの約30%をカバー。
    \item PHP、Mysqlおよび正規表現に基づく。
    \item 模擬ログインを実装し、正規表現マッチングにより必要な情報を入手する。
    \end{tightemize}



%%%%%%%%%%%%%%%%%%%%%%%%%%%%%%%%%%%%%%
%     OPEN SOURCE
%%%%%%%%%%%%%%%%%%%%%%%%%%%%%%%%%%%%%%

\section{コースワーク}

\begin{tabular}{ll}
\href{https://github.com/vinci7/LBMS-by-Django}{\bf LBMS} & Djangoフレームワークに基づく図書館管理システム \\
\href{https://coding.net/u/vinchi/p/DragonCMS}{\bf DragonCMS} & Nodejs、MysqlおよびReactに基づく授業管理システム\\
\end{tabular}

%%%%%%%%%%%%%%%%%%%%%%%%%%%%%%%%%%%%%%
%     AWARDS
%%%%%%%%%%%%%%%%%%%%%%%%%%%%%%%%%%%%%%

\section{受賞経験}

\begin{tabular}{rll}
2016      & 奨学金 & JASSO交流学生奨学金 \\
2016	     & 三等賞  & 浙江大学学術奨学金 \\
2015	     & 三等賞  & 浙江大学学術奨学金 \\
2012	     & 一等賞  & 全国情報学オリンピックコンテスト(NOIP)\\
\end{tabular}


%%%%%%%%%%%%%%%%%%%%%%%%%%%%%%%%%%%%%%
%     LANGUAGE
%%%%%%%%%%%%%%%%%%%%%%%%%%%%%%%%%%%%%%

\section{語学} 

\begin{tabular}{rll}

英語   & ビジネス会話レベル & TOEFL: 84pt \\
日本語	     & 日常会話レベル & \\
中国語	     & ネイティブレベル &  \\
\end{tabular}

%%%%%%%%%%%%%%%%%%%%%%%%%%%%%%%%%%%%%%
%     PUBLICATIONS
%%%%%%%%%%%%%%%%%%%%%%%%%%%%%%%%%%%%%%

% \section{Publications} 
% \renewcommand\refname{\vskip -1.5cm} % Couldn't get this working from the .cls file
% \bibliographystyle{abbrv}
% \bibliography{publications}
% \nocite{*}

\end{minipage} 
\end{document}  \documentclass[]{article}
